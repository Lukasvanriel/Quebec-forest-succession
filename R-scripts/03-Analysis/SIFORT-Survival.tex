% Options for packages loaded elsewhere
\PassOptionsToPackage{unicode}{hyperref}
\PassOptionsToPackage{hyphens}{url}
%
\documentclass[
]{article}
\usepackage{amsmath,amssymb}
\usepackage{iftex}
\ifPDFTeX
  \usepackage[T1]{fontenc}
  \usepackage[utf8]{inputenc}
  \usepackage{textcomp} % provide euro and other symbols
\else % if luatex or xetex
  \usepackage{unicode-math} % this also loads fontspec
  \defaultfontfeatures{Scale=MatchLowercase}
  \defaultfontfeatures[\rmfamily]{Ligatures=TeX,Scale=1}
\fi
\usepackage{lmodern}
\ifPDFTeX\else
  % xetex/luatex font selection
\fi
% Use upquote if available, for straight quotes in verbatim environments
\IfFileExists{upquote.sty}{\usepackage{upquote}}{}
\IfFileExists{microtype.sty}{% use microtype if available
  \usepackage[]{microtype}
  \UseMicrotypeSet[protrusion]{basicmath} % disable protrusion for tt fonts
}{}
\makeatletter
\@ifundefined{KOMAClassName}{% if non-KOMA class
  \IfFileExists{parskip.sty}{%
    \usepackage{parskip}
  }{% else
    \setlength{\parindent}{0pt}
    \setlength{\parskip}{6pt plus 2pt minus 1pt}}
}{% if KOMA class
  \KOMAoptions{parskip=half}}
\makeatother
\usepackage{xcolor}
\usepackage[margin=1in]{geometry}
\usepackage{color}
\usepackage{fancyvrb}
\newcommand{\VerbBar}{|}
\newcommand{\VERB}{\Verb[commandchars=\\\{\}]}
\DefineVerbatimEnvironment{Highlighting}{Verbatim}{commandchars=\\\{\}}
% Add ',fontsize=\small' for more characters per line
\usepackage{framed}
\definecolor{shadecolor}{RGB}{248,248,248}
\newenvironment{Shaded}{\begin{snugshade}}{\end{snugshade}}
\newcommand{\AlertTok}[1]{\textcolor[rgb]{0.94,0.16,0.16}{#1}}
\newcommand{\AnnotationTok}[1]{\textcolor[rgb]{0.56,0.35,0.01}{\textbf{\textit{#1}}}}
\newcommand{\AttributeTok}[1]{\textcolor[rgb]{0.13,0.29,0.53}{#1}}
\newcommand{\BaseNTok}[1]{\textcolor[rgb]{0.00,0.00,0.81}{#1}}
\newcommand{\BuiltInTok}[1]{#1}
\newcommand{\CharTok}[1]{\textcolor[rgb]{0.31,0.60,0.02}{#1}}
\newcommand{\CommentTok}[1]{\textcolor[rgb]{0.56,0.35,0.01}{\textit{#1}}}
\newcommand{\CommentVarTok}[1]{\textcolor[rgb]{0.56,0.35,0.01}{\textbf{\textit{#1}}}}
\newcommand{\ConstantTok}[1]{\textcolor[rgb]{0.56,0.35,0.01}{#1}}
\newcommand{\ControlFlowTok}[1]{\textcolor[rgb]{0.13,0.29,0.53}{\textbf{#1}}}
\newcommand{\DataTypeTok}[1]{\textcolor[rgb]{0.13,0.29,0.53}{#1}}
\newcommand{\DecValTok}[1]{\textcolor[rgb]{0.00,0.00,0.81}{#1}}
\newcommand{\DocumentationTok}[1]{\textcolor[rgb]{0.56,0.35,0.01}{\textbf{\textit{#1}}}}
\newcommand{\ErrorTok}[1]{\textcolor[rgb]{0.64,0.00,0.00}{\textbf{#1}}}
\newcommand{\ExtensionTok}[1]{#1}
\newcommand{\FloatTok}[1]{\textcolor[rgb]{0.00,0.00,0.81}{#1}}
\newcommand{\FunctionTok}[1]{\textcolor[rgb]{0.13,0.29,0.53}{\textbf{#1}}}
\newcommand{\ImportTok}[1]{#1}
\newcommand{\InformationTok}[1]{\textcolor[rgb]{0.56,0.35,0.01}{\textbf{\textit{#1}}}}
\newcommand{\KeywordTok}[1]{\textcolor[rgb]{0.13,0.29,0.53}{\textbf{#1}}}
\newcommand{\NormalTok}[1]{#1}
\newcommand{\OperatorTok}[1]{\textcolor[rgb]{0.81,0.36,0.00}{\textbf{#1}}}
\newcommand{\OtherTok}[1]{\textcolor[rgb]{0.56,0.35,0.01}{#1}}
\newcommand{\PreprocessorTok}[1]{\textcolor[rgb]{0.56,0.35,0.01}{\textit{#1}}}
\newcommand{\RegionMarkerTok}[1]{#1}
\newcommand{\SpecialCharTok}[1]{\textcolor[rgb]{0.81,0.36,0.00}{\textbf{#1}}}
\newcommand{\SpecialStringTok}[1]{\textcolor[rgb]{0.31,0.60,0.02}{#1}}
\newcommand{\StringTok}[1]{\textcolor[rgb]{0.31,0.60,0.02}{#1}}
\newcommand{\VariableTok}[1]{\textcolor[rgb]{0.00,0.00,0.00}{#1}}
\newcommand{\VerbatimStringTok}[1]{\textcolor[rgb]{0.31,0.60,0.02}{#1}}
\newcommand{\WarningTok}[1]{\textcolor[rgb]{0.56,0.35,0.01}{\textbf{\textit{#1}}}}
\usepackage{graphicx}
\makeatletter
\def\maxwidth{\ifdim\Gin@nat@width>\linewidth\linewidth\else\Gin@nat@width\fi}
\def\maxheight{\ifdim\Gin@nat@height>\textheight\textheight\else\Gin@nat@height\fi}
\makeatother
% Scale images if necessary, so that they will not overflow the page
% margins by default, and it is still possible to overwrite the defaults
% using explicit options in \includegraphics[width, height, ...]{}
\setkeys{Gin}{width=\maxwidth,height=\maxheight,keepaspectratio}
% Set default figure placement to htbp
\makeatletter
\def\fps@figure{htbp}
\makeatother
\setlength{\emergencystretch}{3em} % prevent overfull lines
\providecommand{\tightlist}{%
  \setlength{\itemsep}{0pt}\setlength{\parskip}{0pt}}
\setcounter{secnumdepth}{-\maxdimen} % remove section numbering
\ifLuaTeX
  \usepackage{selnolig}  % disable illegal ligatures
\fi
\usepackage{bookmark}
\IfFileExists{xurl.sty}{\usepackage{xurl}}{} % add URL line breaks if available
\urlstyle{same}
\hypersetup{
  pdftitle={Transition probabilities from survival hazards},
  pdfauthor={Lukas Van Riel},
  hidelinks,
  pdfcreator={LaTeX via pandoc}}

\title{Transition probabilities from survival hazards}
\author{Lukas Van Riel}
\date{}

\begin{document}
\maketitle

\paragraph{\texorpdfstring{Problem: How to use the INLAjoint output
hazards to compute and visualise transition probabilities between
classes?\\
}{Problem: How to use the INLAjoint output hazards to compute and visualise transition probabilities between classes? }}\label{problem-how-to-use-the-inlajoint-output-hazards-to-compute-and-visualise-transition-probabilities-between-classes}

First load the required packages.

And the functions we will need.

Then import the data, either the complete INLAjoint object (the complete
model output), or the object already reworked into a useful dataframe.
Make sure the allj.T.C.Pc folder and full\_df.RDS file are located in
the same folder as this document or change the directories.

\begin{Shaded}
\begin{Highlighting}[]
\CommentTok{\# Set directory to be where this document is located}
\NormalTok{here}\SpecialCharTok{::}\FunctionTok{i\_am}\NormalTok{(}\StringTok{"SIFORT{-}Survival.Rmd"}\NormalTok{)}
\end{Highlighting}
\end{Shaded}

\begin{verbatim}
## here() starts at /Users/lukas/Desktop/RProjects/Quebec-forest-succession/R-scripts/03-Analysis
\end{verbatim}

\begin{Shaded}
\begin{Highlighting}[]
\CommentTok{\# Create original output object}
\CommentTok{\#inla.allj.T.C.Pc \textless{}{-} set.back.inla("allj.T.C.Pc")}

\CommentTok{\# Load output parameter data as a dataframe, where the parameter, including CI, values are found in the last columns. }
\NormalTok{full.df }\OtherTok{\textless{}{-}} \FunctionTok{readRDS}\NormalTok{(}\FunctionTok{here}\NormalTok{(}\StringTok{"full\_df.RDS"}\NormalTok{))}
\CommentTok{\# Specific values can now be extracted as follows (e.g. median alpha value for transition 1{-}\textgreater{}2):}
\NormalTok{full.df}\SpecialCharTok{$}\NormalTok{alpha[[}\DecValTok{1}\NormalTok{]]}\SpecialCharTok{$}\StringTok{"0.5quant"}
\end{Highlighting}
\end{Shaded}

\begin{verbatim}
## [1] 2.192838
\end{verbatim}

Let's create an object to easily access the names of the classes and
transitions.

\begin{Shaded}
\begin{Highlighting}[]
\NormalTok{info.classes }\OtherTok{\textless{}{-}} \FunctionTok{data.frame}\NormalTok{(}\AttributeTok{Nb=}\FunctionTok{seq}\NormalTok{(}\DecValTok{1}\SpecialCharTok{:}\DecValTok{9}\NormalTok{), }\AttributeTok{Name=}\FunctionTok{c}\NormalTok{(}\StringTok{"Paper birch"}\NormalTok{, }\StringTok{"Shade int."}\NormalTok{, }\StringTok{"Yellow birch"}\NormalTok{, }\StringTok{"Maples"}\NormalTok{, }\StringTok{"Other dec."}\NormalTok{, }\StringTok{"Bals. fir"}\NormalTok{, }\StringTok{"B/R spruce"}\NormalTok{, }\StringTok{"Jack pine"}\NormalTok{, }\StringTok{"Other con."}\NormalTok{))}
\end{Highlighting}
\end{Shaded}

We can take a look at the parameter values and their credible intervals:

\begin{Shaded}
\begin{Highlighting}[]
\CommentTok{\# You can change the class where the transitions start and the parameter you want to see.}
\FunctionTok{plot\_inla\_param}\NormalTok{(full.df, info.classes, }\AttributeTok{fr =} \DecValTok{1}\NormalTok{, }\AttributeTok{parameter =} \StringTok{"cov\_Tmean"}\NormalTok{)}
\end{Highlighting}
\end{Shaded}

\begin{verbatim}
## Warning: Using an external vector in selections was deprecated in tidyselect 1.1.0.
## i Please use `all_of()` or `any_of()` instead.
##   # Was:
##   data %>% select(parameter)
## 
##   # Now:
##   data %>% select(all_of(parameter))
## 
## See <https://tidyselect.r-lib.org/reference/faq-external-vector.html>.
## This warning is displayed once every 8 hours.
## Call `lifecycle::last_lifecycle_warnings()` to see where this warning was
## generated.
\end{verbatim}

\includegraphics{SIFORT-Survival_files/figure-latex/unnamed-chunk-5-1.pdf}

Alternatively we can plot all parameters simultaneously:

\begin{Shaded}
\begin{Highlighting}[]
\FunctionTok{plot\_inla\_covariates}\NormalTok{(full.df, info.classes, }\AttributeTok{fr =} \DecValTok{7}\NormalTok{)}
\end{Highlighting}
\end{Shaded}

\includegraphics{SIFORT-Survival_files/figure-latex/unnamed-chunk-6-1.pdf}

Let's now try to plot the transition probabilities over time. I have
included all the functions that are used for this in the next chunks
rather than at the top. I think this might make it easier to check the
functions if you want to. First, we'll define the Weibull hazard
function and the cumulative Weibull hazard:

\begin{Shaded}
\begin{Highlighting}[]
\CommentTok{\# Weibull hazard. Covariate effects can be included in the lambda factor}
\NormalTok{Wb }\OtherTok{\textless{}{-}} \ControlFlowTok{function}\NormalTok{(t, alpha, lambda) \{}
\NormalTok{  alpha }\SpecialCharTok{*}\NormalTok{ t}\SpecialCharTok{**}\NormalTok{(alpha}\DecValTok{{-}1}\NormalTok{) }\SpecialCharTok{*}\NormalTok{ lambda}
\NormalTok{\}}
\CommentTok{\# Cumulative Weibull hazard, the integrated Weibull hazard.}
\NormalTok{Wb\_H }\OtherTok{\textless{}{-}} \ControlFlowTok{function}\NormalTok{(t, alpha, lambda) \{}
\NormalTok{  t}\SpecialCharTok{**}\NormalTok{(alpha) }\SpecialCharTok{*}\NormalTok{ lambda}
\NormalTok{\}}
\end{Highlighting}
\end{Shaded}

Next, we'll use the Weibull function to compute the hazards for times 0
up to 60 years (can be altered). In Alvares et al.~(2022) they call this
risks, but really they are the hazards. I used the name risks since I
was following along with the paper (section 7.4).

\begin{Shaded}
\begin{Highlighting}[]
\CommentTok{\# Computes the risks for the baseline hazards and including the effects of the perturbation type (1=fire, 2=harvest, 3=pest).}
\NormalTok{get\_risks }\OtherTok{\textless{}{-}} \ControlFlowTok{function}\NormalTok{(inla.output, }\AttributeTok{t=}\DecValTok{50}\NormalTok{)\{}
\NormalTok{  years }\OtherTok{\textless{}{-}} \FunctionTok{seq}\NormalTok{(}\DecValTok{0}\NormalTok{, t, }\AttributeTok{by =} \DecValTok{1}\NormalTok{)}
\NormalTok{  risks }\OtherTok{\textless{}{-}} \FunctionTok{list}\NormalTok{()}
  \ControlFlowTok{for}\NormalTok{(i }\ControlFlowTok{in} \DecValTok{1}\SpecialCharTok{:}\FunctionTok{nrow}\NormalTok{(inla.output)) \{}
\NormalTok{    risks[[i]] }\OtherTok{\textless{}{-}} \FunctionTok{list}\NormalTok{()}
\NormalTok{    risks[[i]]}\SpecialCharTok{$}\NormalTok{baseline }\OtherTok{\textless{}{-}} \FunctionTok{Wb}\NormalTok{(years, inla.output}\SpecialCharTok{$}\NormalTok{alpha[[i]]}\SpecialCharTok{$}\NormalTok{mean, }
                              \FunctionTok{exp}\NormalTok{(inla.output}\SpecialCharTok{$}\NormalTok{Intercept[[i]]}\SpecialCharTok{$}\NormalTok{mean))}
\NormalTok{    risks[[i]]}\SpecialCharTok{$}\NormalTok{fire }\OtherTok{\textless{}{-}} \FunctionTok{Wb}\NormalTok{(years, inla.output}\SpecialCharTok{$}\NormalTok{alpha[[i]]}\SpecialCharTok{$}\NormalTok{mean, }
                          \FunctionTok{exp}\NormalTok{(inla.output}\SpecialCharTok{$}\NormalTok{Intercept[[i]]}\SpecialCharTok{$}\NormalTok{mean }\SpecialCharTok{+}\NormalTok{ full.df}\SpecialCharTok{$}\NormalTok{cov\_pert\_class1[[i]]}\SpecialCharTok{$}\NormalTok{mean))}
\NormalTok{    risks[[i]]}\SpecialCharTok{$}\NormalTok{harvest }\OtherTok{\textless{}{-}} \FunctionTok{Wb}\NormalTok{(years, inla.output}\SpecialCharTok{$}\NormalTok{alpha[[i]]}\SpecialCharTok{$}\NormalTok{mean, }
                             \FunctionTok{exp}\NormalTok{(inla.output}\SpecialCharTok{$}\NormalTok{Intercept[[i]]}\SpecialCharTok{$}\NormalTok{mean }\SpecialCharTok{+}\NormalTok{ full.df}\SpecialCharTok{$}\NormalTok{cov\_pert\_class2[[i]]}\SpecialCharTok{$}\NormalTok{mean))}
\NormalTok{    risks[[i]]}\SpecialCharTok{$}\NormalTok{pest }\OtherTok{\textless{}{-}} \FunctionTok{Wb}\NormalTok{(years, inla.output}\SpecialCharTok{$}\NormalTok{alpha[[i]]}\SpecialCharTok{$}\NormalTok{mean, }
                          \FunctionTok{exp}\NormalTok{(inla.output}\SpecialCharTok{$}\NormalTok{Intercept[[i]]}\SpecialCharTok{$}\NormalTok{mean }\SpecialCharTok{+}\NormalTok{ full.df}\SpecialCharTok{$}\NormalTok{cov\_pert\_class3[[i]]}\SpecialCharTok{$}\NormalTok{mean))}
\NormalTok{  \}}
  \FunctionTok{names}\NormalTok{(risks) }\OtherTok{\textless{}{-}} \FunctionTok{as.character}\NormalTok{(}\DecValTok{1}\SpecialCharTok{:}\FunctionTok{nrow}\NormalTok{(inla.output))}
\NormalTok{  risks}
\NormalTok{\}}

\NormalTok{risks }\OtherTok{\textless{}{-}} \FunctionTok{get\_risks}\NormalTok{(full.df, }\AttributeTok{t =} \DecValTok{60}\NormalTok{)}
\end{Highlighting}
\end{Shaded}

Now I wrote 2 functions to calculate the probabilities to stay in a
certain state as well as the transition probabilities:

\begin{Shaded}
\begin{Highlighting}[]
\CommentTok{\#Functions based on the formulas used in section 7.4 of Alvares et al. (2022). The structure of the functions are similar. They still need to be exponentiated in the case of the P(i{-}\textgreater{}i). Cumsum is used (I think) to approximate integrals.}
\NormalTok{trans\_prob\_timelapse\_stay }\OtherTok{\textless{}{-}} \ControlFlowTok{function}\NormalTok{(inla.output, state, }\AttributeTok{t=}\DecValTok{50}\NormalTok{, }\AttributeTok{risks=}\ConstantTok{NULL}\NormalTok{) \{}
  \CommentTok{\# Calculate the risks if they are not included or do not match the requested times}
  \ControlFlowTok{if}\NormalTok{(}\FunctionTok{is.null}\NormalTok{(risks) }\SpecialCharTok{||} \FunctionTok{length}\NormalTok{(risks[[}\DecValTok{1}\NormalTok{]]}\SpecialCharTok{$}\NormalTok{baseline) }\SpecialCharTok{!=}\NormalTok{ (t}\SpecialCharTok{+}\DecValTok{1}\NormalTok{)) risks }\OtherTok{\textless{}{-}} \FunctionTok{get\_risks}\NormalTok{(inla.output, t)}
  
  \CommentTok{\#See which transitions depart from the state in question}
\NormalTok{  trans.out }\OtherTok{\textless{}{-}}\NormalTok{ inla.output }\SpecialCharTok{\%\textgreater{}\%} 
    \FunctionTok{filter}\NormalTok{(from }\SpecialCharTok{==}\NormalTok{ state) }\SpecialCharTok{\%\textgreater{}\%} 
    \FunctionTok{pull}\NormalTok{(inla\_trans)}
  
\NormalTok{  stay.risks }\OtherTok{\textless{}{-}} \FunctionTok{list}\NormalTok{()}
  
  \CommentTok{\#apply the formula of Alvares et al. (2022)}
\NormalTok{  cumulatives }\OtherTok{\textless{}{-}} \FunctionTok{sapply}\NormalTok{(trans.out, }\AttributeTok{FUN=}\ControlFlowTok{function}\NormalTok{(x) }\FunctionTok{cumsum}\NormalTok{(risks[[x]]}\SpecialCharTok{$}\NormalTok{baseline))}
\NormalTok{  stay.risks}\SpecialCharTok{$}\NormalTok{baseline }\OtherTok{\textless{}{-}} \FunctionTok{apply}\NormalTok{(}\FunctionTok{as.matrix}\NormalTok{(cumulatives), }\DecValTok{1}\NormalTok{, sum)}
\NormalTok{  cumulatives }\OtherTok{\textless{}{-}} \FunctionTok{sapply}\NormalTok{(trans.out, }\AttributeTok{FUN=}\ControlFlowTok{function}\NormalTok{(x) }\FunctionTok{cumsum}\NormalTok{(risks[[x]]}\SpecialCharTok{$}\NormalTok{fire))}
\NormalTok{  stay.risks}\SpecialCharTok{$}\NormalTok{fire }\OtherTok{\textless{}{-}} \FunctionTok{apply}\NormalTok{(}\FunctionTok{as.matrix}\NormalTok{(cumulatives), }\DecValTok{1}\NormalTok{, sum)}
\NormalTok{  cumulatives }\OtherTok{\textless{}{-}} \FunctionTok{sapply}\NormalTok{(trans.out, }\AttributeTok{FUN=}\ControlFlowTok{function}\NormalTok{(x) }\FunctionTok{cumsum}\NormalTok{(risks[[x]]}\SpecialCharTok{$}\NormalTok{harvest))}
\NormalTok{  stay.risks}\SpecialCharTok{$}\NormalTok{harvest }\OtherTok{\textless{}{-}} \FunctionTok{apply}\NormalTok{(}\FunctionTok{as.matrix}\NormalTok{(cumulatives), }\DecValTok{1}\NormalTok{, sum)}
\NormalTok{  cumulatives }\OtherTok{\textless{}{-}} \FunctionTok{sapply}\NormalTok{(trans.out, }\AttributeTok{FUN=}\ControlFlowTok{function}\NormalTok{(x) }\FunctionTok{cumsum}\NormalTok{(risks[[x]]}\SpecialCharTok{$}\NormalTok{pest))}
\NormalTok{  stay.risks}\SpecialCharTok{$}\NormalTok{pest }\OtherTok{\textless{}{-}} \FunctionTok{apply}\NormalTok{(}\FunctionTok{as.matrix}\NormalTok{(cumulatives), }\DecValTok{1}\NormalTok{, sum)}
  
\NormalTok{  stay.risks}
\NormalTok{\}}

\NormalTok{trans\_prob\_timelapse\_leave }\OtherTok{\textless{}{-}} \ControlFlowTok{function}\NormalTok{(inla.output, state.fr, state.to, }\AttributeTok{t=}\DecValTok{50}\NormalTok{, }\AttributeTok{risks=}\ConstantTok{NULL}\NormalTok{) \{}
  \ControlFlowTok{if}\NormalTok{(}\FunctionTok{is.null}\NormalTok{(risks) }\SpecialCharTok{||} \FunctionTok{length}\NormalTok{(risks[[}\DecValTok{1}\NormalTok{]]}\SpecialCharTok{$}\NormalTok{baseline) }\SpecialCharTok{!=}\NormalTok{ (t}\SpecialCharTok{+}\DecValTok{1}\NormalTok{)) risks }\OtherTok{\textless{}{-}} \FunctionTok{get\_risks}\NormalTok{(inla.output, t)}
  
\NormalTok{  trans.number }\OtherTok{\textless{}{-}}\NormalTok{ full.df }\SpecialCharTok{\%\textgreater{}\%} 
    \FunctionTok{filter}\NormalTok{(from }\SpecialCharTok{==}\NormalTok{ state.fr, to }\SpecialCharTok{==}\NormalTok{ state.to) }\SpecialCharTok{\%\textgreater{}\%} 
    \FunctionTok{pull}\NormalTok{(inla\_trans)}
  
\NormalTok{  leave.risks }\OtherTok{\textless{}{-}} \FunctionTok{list}\NormalTok{()}
  
\NormalTok{  leave.risks}\SpecialCharTok{$}\NormalTok{baseline }\OtherTok{\textless{}{-}} \FunctionTok{cumsum}\NormalTok{(}\FunctionTok{exp}\NormalTok{(}\SpecialCharTok{{-}}\FunctionTok{trans\_prob\_timelapse\_stay}\NormalTok{(inla.output, state.fr, }\AttributeTok{t=}\NormalTok{t, }\AttributeTok{risks=}\NormalTok{risks)}\SpecialCharTok{$}\NormalTok{baseline) }\SpecialCharTok{*}\NormalTok{ risks[[trans.number]]}\SpecialCharTok{$}\NormalTok{baseline }\SpecialCharTok{*} \FunctionTok{exp}\NormalTok{(}\SpecialCharTok{{-}}\FunctionTok{trans\_prob\_timelapse\_stay}\NormalTok{(inla.output, state.to, }\AttributeTok{t=}\NormalTok{t, }\AttributeTok{risks=}\NormalTok{risks)}\SpecialCharTok{$}\NormalTok{baseline))}
  
\NormalTok{  leave.risks}\SpecialCharTok{$}\NormalTok{fire }\OtherTok{\textless{}{-}} \FunctionTok{cumsum}\NormalTok{(}\FunctionTok{exp}\NormalTok{(}\SpecialCharTok{{-}}\FunctionTok{trans\_prob\_timelapse\_stay}\NormalTok{(inla.output, state.fr, }\AttributeTok{t=}\NormalTok{t, }\AttributeTok{risks=}\NormalTok{risks)}\SpecialCharTok{$}\NormalTok{fire) }\SpecialCharTok{*}\NormalTok{ risks[[trans.number]]}\SpecialCharTok{$}\NormalTok{fire }\SpecialCharTok{*} \FunctionTok{exp}\NormalTok{(}\SpecialCharTok{{-}}\FunctionTok{trans\_prob\_timelapse\_stay}\NormalTok{(inla.output, state.to, }\AttributeTok{t=}\NormalTok{t, }\AttributeTok{risks=}\NormalTok{risks)}\SpecialCharTok{$}\NormalTok{fire))}
\NormalTok{  leave.risks}\SpecialCharTok{$}\NormalTok{harvest }\OtherTok{\textless{}{-}} \FunctionTok{cumsum}\NormalTok{(}\FunctionTok{exp}\NormalTok{(}\SpecialCharTok{{-}}\FunctionTok{trans\_prob\_timelapse\_stay}\NormalTok{(inla.output, state.fr, }\AttributeTok{t=}\NormalTok{t, }\AttributeTok{risks=}\NormalTok{risks)}\SpecialCharTok{$}\NormalTok{harvest) }\SpecialCharTok{*}\NormalTok{ risks[[trans.number]]}\SpecialCharTok{$}\NormalTok{harvest }\SpecialCharTok{*} \FunctionTok{exp}\NormalTok{(}\SpecialCharTok{{-}}\FunctionTok{trans\_prob\_timelapse\_stay}\NormalTok{(inla.output, state.to, }\AttributeTok{t=}\NormalTok{t, }\AttributeTok{risks=}\NormalTok{risks)}\SpecialCharTok{$}\NormalTok{harvest))}
\NormalTok{  leave.risks}\SpecialCharTok{$}\NormalTok{pest }\OtherTok{\textless{}{-}} \FunctionTok{cumsum}\NormalTok{(}\FunctionTok{exp}\NormalTok{(}\SpecialCharTok{{-}}\FunctionTok{trans\_prob\_timelapse\_stay}\NormalTok{(inla.output, state.fr, }\AttributeTok{t=}\NormalTok{t, }\AttributeTok{risks=}\NormalTok{risks)}\SpecialCharTok{$}\NormalTok{pest) }\SpecialCharTok{*}\NormalTok{ risks[[trans.number]]}\SpecialCharTok{$}\NormalTok{pest }\SpecialCharTok{*} \FunctionTok{exp}\NormalTok{(}\SpecialCharTok{{-}}\FunctionTok{trans\_prob\_timelapse\_stay}\NormalTok{(inla.output, state.to, }\AttributeTok{t=}\NormalTok{t, }\AttributeTok{risks=}\NormalTok{risks)}\SpecialCharTok{$}\NormalTok{pest))}
  
\NormalTok{  leave.risks}
\NormalTok{\}}
\end{Highlighting}
\end{Shaded}

We can now use these functions to try and calculate the transition
probabilities. To get the probabilities of staying in a state, the
results still need to be exponentiated. For the transition probabilities
of changing state, it seems this is not the case based on the equations
in section 7.4.

\begin{Shaded}
\begin{Highlighting}[]
\CommentTok{\# Lets plot the probability over time of staying in a certain state, as function of perturbation class (black=baseline, red=fire, green=harvest, purple=pest).}
\NormalTok{years}\OtherTok{=}\DecValTok{0}\SpecialCharTok{:}\DecValTok{60}
\NormalTok{fr }\OtherTok{=} \DecValTok{1} \CommentTok{\# Which state are we interested in?}
\NormalTok{prob.stay }\OtherTok{\textless{}{-}} \FunctionTok{trans\_prob\_timelapse\_stay}\NormalTok{(full.df, }\AttributeTok{state =}\NormalTok{ fr, }\AttributeTok{t=}\DecValTok{60}\NormalTok{)}
\FunctionTok{plot}\NormalTok{(years, }\FunctionTok{exp}\NormalTok{(}\SpecialCharTok{{-}}\NormalTok{prob.stay}\SpecialCharTok{$}\NormalTok{baseline), }\AttributeTok{pch=}\DecValTok{20}\NormalTok{, }\AttributeTok{cex =}\DecValTok{1}\NormalTok{, }\AttributeTok{col=}\StringTok{"black"}\NormalTok{, }\AttributeTok{ylab=}\StringTok{"Transition probability"}\NormalTok{, }\AttributeTok{xlab=}\StringTok{"time [years]"}\NormalTok{, }\AttributeTok{main=}\FunctionTok{paste}\NormalTok{(}\StringTok{"From"}\NormalTok{, info.classes}\SpecialCharTok{$}\NormalTok{Name[fr]))}
\FunctionTok{points}\NormalTok{(years, }\FunctionTok{exp}\NormalTok{(}\SpecialCharTok{{-}}\NormalTok{prob.stay}\SpecialCharTok{$}\NormalTok{fire), }\AttributeTok{pch=}\DecValTok{20}\NormalTok{, }\AttributeTok{cex =}\DecValTok{1}\NormalTok{, }\AttributeTok{col=}\StringTok{"red"}\NormalTok{)}
\FunctionTok{points}\NormalTok{(years, }\FunctionTok{exp}\NormalTok{(}\SpecialCharTok{{-}}\NormalTok{prob.stay}\SpecialCharTok{$}\NormalTok{harvest), }\AttributeTok{pch=}\DecValTok{20}\NormalTok{, }\AttributeTok{cex =}\DecValTok{1}\NormalTok{, }\AttributeTok{col=}\StringTok{"darkgreen"}\NormalTok{)}
\FunctionTok{points}\NormalTok{(years, }\FunctionTok{exp}\NormalTok{(}\SpecialCharTok{{-}}\NormalTok{prob.stay}\SpecialCharTok{$}\NormalTok{pest), }\AttributeTok{pch=}\DecValTok{20}\NormalTok{, }\AttributeTok{cex =}\DecValTok{1}\NormalTok{, }\AttributeTok{col=}\StringTok{"purple"}\NormalTok{)}
\end{Highlighting}
\end{Shaded}

\includegraphics{SIFORT-Survival_files/figure-latex/unnamed-chunk-10-1.pdf}

Which might in theory be possible, but checking other states reveals
that the probabilities all go to 0 at some point, some even very
quickly:

\begin{Shaded}
\begin{Highlighting}[]
\NormalTok{fr }\OtherTok{=} \DecValTok{6} \CommentTok{\# Which state are we interested in?}
\NormalTok{prob.stay }\OtherTok{\textless{}{-}} \FunctionTok{trans\_prob\_timelapse\_stay}\NormalTok{(full.df, }\AttributeTok{state =}\NormalTok{ fr, }\AttributeTok{t=}\DecValTok{60}\NormalTok{)}
\FunctionTok{plot}\NormalTok{(years, }\FunctionTok{exp}\NormalTok{(}\SpecialCharTok{{-}}\NormalTok{prob.stay}\SpecialCharTok{$}\NormalTok{baseline), }\AttributeTok{pch=}\DecValTok{20}\NormalTok{, }\AttributeTok{cex =}\DecValTok{1}\NormalTok{, }\AttributeTok{col=}\StringTok{"black"}\NormalTok{, }\AttributeTok{ylab=}\StringTok{"Transition probability"}\NormalTok{, }\AttributeTok{xlab=}\StringTok{"time [years]"}\NormalTok{, }\AttributeTok{main=}\FunctionTok{paste}\NormalTok{(}\StringTok{"From"}\NormalTok{, info.classes}\SpecialCharTok{$}\NormalTok{Name[fr]))}
\FunctionTok{points}\NormalTok{(years, }\FunctionTok{exp}\NormalTok{(}\SpecialCharTok{{-}}\NormalTok{prob.stay}\SpecialCharTok{$}\NormalTok{fire), }\AttributeTok{pch=}\DecValTok{20}\NormalTok{, }\AttributeTok{cex =}\DecValTok{1}\NormalTok{, }\AttributeTok{col=}\StringTok{"red"}\NormalTok{)}
\FunctionTok{points}\NormalTok{(years, }\FunctionTok{exp}\NormalTok{(}\SpecialCharTok{{-}}\NormalTok{prob.stay}\SpecialCharTok{$}\NormalTok{harvest), }\AttributeTok{pch=}\DecValTok{20}\NormalTok{, }\AttributeTok{cex =}\DecValTok{1}\NormalTok{, }\AttributeTok{col=}\StringTok{"darkgreen"}\NormalTok{)}
\FunctionTok{points}\NormalTok{(years, }\FunctionTok{exp}\NormalTok{(}\SpecialCharTok{{-}}\NormalTok{prob.stay}\SpecialCharTok{$}\NormalTok{pest), }\AttributeTok{pch=}\DecValTok{20}\NormalTok{, }\AttributeTok{cex =}\DecValTok{1}\NormalTok{, }\AttributeTok{col=}\StringTok{"purple"}\NormalTok{)}
\end{Highlighting}
\end{Shaded}

\includegraphics{SIFORT-Survival_files/figure-latex/unnamed-chunk-11-1.pdf}

Now let's check the probabilities to change state, again by perturbation
type. According to the equations in section 7.4, they do not need to be
exponentiated.

\begin{Shaded}
\begin{Highlighting}[]
\NormalTok{fr}\OtherTok{=}\DecValTok{1}
\NormalTok{to}\OtherTok{=}\DecValTok{2}
\NormalTok{prob.leave }\OtherTok{\textless{}{-}} \FunctionTok{trans\_prob\_timelapse\_leave}\NormalTok{(full.df, fr, to, }\AttributeTok{t=}\DecValTok{60}\NormalTok{)}
\FunctionTok{plot}\NormalTok{(years, prob.leave}\SpecialCharTok{$}\NormalTok{baseline, }\AttributeTok{pch=}\DecValTok{20}\NormalTok{, }\AttributeTok{cex =}\DecValTok{1}\NormalTok{, }\AttributeTok{col=}\StringTok{"black"}\NormalTok{, }\AttributeTok{ylim=}\FunctionTok{c}\NormalTok{(}\DecValTok{0}\NormalTok{,}\FloatTok{0.2}\NormalTok{), }\AttributeTok{ylab=}\StringTok{"Transition probability"}\NormalTok{, }\AttributeTok{xlab=}\StringTok{"time [years]"}\NormalTok{, }\AttributeTok{main=}\FunctionTok{paste}\NormalTok{(info.classes}\SpecialCharTok{$}\NormalTok{Name[fr], }\StringTok{"to"}\NormalTok{, info.classes}\SpecialCharTok{$}\NormalTok{Name[to]))}
\FunctionTok{points}\NormalTok{(years, prob.leave}\SpecialCharTok{$}\NormalTok{fire, }\AttributeTok{pch=}\DecValTok{20}\NormalTok{, }\AttributeTok{cex =}\DecValTok{1}\NormalTok{, }\AttributeTok{col=}\StringTok{"red"}\NormalTok{)}
\FunctionTok{points}\NormalTok{(years, prob.leave}\SpecialCharTok{$}\NormalTok{harvest, }\AttributeTok{pch=}\DecValTok{20}\NormalTok{, }\AttributeTok{cex =}\DecValTok{1}\NormalTok{, }\AttributeTok{col=}\StringTok{"darkgreen"}\NormalTok{)}
\FunctionTok{points}\NormalTok{(years, prob.leave}\SpecialCharTok{$}\NormalTok{pest, }\AttributeTok{pch=}\DecValTok{20}\NormalTok{, }\AttributeTok{cex =}\DecValTok{1}\NormalTok{, }\AttributeTok{col=}\StringTok{"purple"}\NormalTok{)}
\end{Highlighting}
\end{Shaded}

\includegraphics{SIFORT-Survival_files/figure-latex/unnamed-chunk-12-1.pdf}

Let's look at different transitions over time, but starting from the
same state. Black dots are the probability to stay in the state while
red are the different probabilities of transition towards another state.

\begin{Shaded}
\begin{Highlighting}[]
\NormalTok{fr }\OtherTok{=} \DecValTok{1}
\FunctionTok{plot}\NormalTok{(years, }\FunctionTok{exp}\NormalTok{(}\SpecialCharTok{{-}}\FunctionTok{trans\_prob\_timelapse\_stay}\NormalTok{(full.df, }\AttributeTok{state =}\NormalTok{ fr, }\AttributeTok{t=}\DecValTok{60}\NormalTok{)}\SpecialCharTok{$}\NormalTok{baseline), }\AttributeTok{pch=}\DecValTok{20}\NormalTok{, }\AttributeTok{cex =}\DecValTok{1}\NormalTok{, }\AttributeTok{col=}\StringTok{"black"}\NormalTok{, }\AttributeTok{ylab=}\StringTok{"Transition probability"}\NormalTok{, }\AttributeTok{xlab=}\StringTok{"time [years]"}\NormalTok{, }\AttributeTok{main=}\FunctionTok{paste}\NormalTok{(}\StringTok{"From"}\NormalTok{, info.classes}\SpecialCharTok{$}\NormalTok{Name[fr]))}
\ControlFlowTok{for}\NormalTok{(j }\ControlFlowTok{in}\NormalTok{ (}\FunctionTok{seq}\NormalTok{(}\DecValTok{1}\NormalTok{,}\DecValTok{9}\NormalTok{, }\AttributeTok{by=}\DecValTok{1}\NormalTok{)[}\SpecialCharTok{{-}}\NormalTok{fr])) \{}
  \FunctionTok{points}\NormalTok{(years, }\FunctionTok{trans\_prob\_timelapse\_leave}\NormalTok{(full.df, fr, j, }\AttributeTok{t=}\DecValTok{60}\NormalTok{)}\SpecialCharTok{$}\NormalTok{baseline, }\AttributeTok{pch=}\DecValTok{20}\NormalTok{, }\AttributeTok{cex =}\DecValTok{1}\NormalTok{, }\AttributeTok{col=}\StringTok{"red"}\NormalTok{)}
\NormalTok{\}}
\end{Highlighting}
\end{Shaded}

\includegraphics{SIFORT-Survival_files/figure-latex/unnamed-chunk-13-1.pdf}

Let's try to create a transition probability matrix now, for t=10 years.
Rows indicate departure state, columns destination (so element {[}1,2{]}
is the probability of transition from state 1 to state 2 after 10
years). At all times, adding the probabilities by row should result in
1. Let's see if that is the case:

\begin{Shaded}
\begin{Highlighting}[]
\NormalTok{p.matr }\OtherTok{\textless{}{-}} \FunctionTok{matrix}\NormalTok{(}\DecValTok{0}\NormalTok{, }\DecValTok{9}\NormalTok{, }\DecValTok{9}\NormalTok{)}
\NormalTok{risks }\OtherTok{\textless{}{-}} \FunctionTok{get\_risks}\NormalTok{(full.df, }\AttributeTok{t=}\DecValTok{50}\NormalTok{)}
\CommentTok{\# We can create a probability matrix for each t. Lets pick t=10 years, so this is element 11 in each vector}
\ControlFlowTok{for}\NormalTok{(i }\ControlFlowTok{in} \DecValTok{1}\SpecialCharTok{:}\FunctionTok{nrow}\NormalTok{(p.matr))\{}
  \ControlFlowTok{for}\NormalTok{(j }\ControlFlowTok{in} \DecValTok{1}\SpecialCharTok{:}\FunctionTok{ncol}\NormalTok{(p.matr))\{}
    \ControlFlowTok{if}\NormalTok{(i }\SpecialCharTok{==}\NormalTok{ j) \{p.matr[i,j] }\OtherTok{\textless{}{-}} \FunctionTok{exp}\NormalTok{(}\SpecialCharTok{{-}}\FunctionTok{trans\_prob\_timelapse\_stay}\NormalTok{(full.df, i, }\AttributeTok{t=}\DecValTok{60}\NormalTok{, risks)}\SpecialCharTok{$}\NormalTok{baseline)[}\DecValTok{11}\NormalTok{]}
\NormalTok{    \} }\ControlFlowTok{else}\NormalTok{ \{p.matr[i,j] }\OtherTok{\textless{}{-}} \FunctionTok{trans\_prob\_timelapse\_leave}\NormalTok{(full.df, i, j, }\AttributeTok{t=}\DecValTok{60}\NormalTok{, risks)}\SpecialCharTok{$}\NormalTok{baseline[}\DecValTok{11}\NormalTok{]\}}
\NormalTok{  \}}
\NormalTok{\}}
\CommentTok{\#heatmap(p.matr, symm = T)}

\FunctionTok{apply}\NormalTok{(p.matr, }\DecValTok{1}\NormalTok{, sum)}
\end{Highlighting}
\end{Shaded}

\begin{verbatim}
## [1] 0.9719050 0.9348895 0.8260202 0.9176409 0.9715195 0.7040014 0.9412288
## [8] 0.9745535 0.7733566
\end{verbatim}

Which does not sum to 1, something must have gone wrong. It is not
entirely unexpected since we never explicitly enforced the 1-(sum of all
other probabilities) like they do in their section 7.4 (for p13 and
p23). However, doing it the way they do seems arbitrary: Why don't they
calculate the p13 and p23 using the formula while using the 1-(sum) for
another probability. It just seems there are formula's for each
probability, so it is unclear where to make sure the probabliities sum
to 1. Maybe if we take the exponential?

\begin{Shaded}
\begin{Highlighting}[]
\NormalTok{p.matr }\OtherTok{\textless{}{-}} \FunctionTok{matrix}\NormalTok{(}\DecValTok{0}\NormalTok{, }\DecValTok{9}\NormalTok{, }\DecValTok{9}\NormalTok{)}
\NormalTok{risks }\OtherTok{\textless{}{-}} \FunctionTok{get\_risks}\NormalTok{(full.df, }\AttributeTok{t=}\DecValTok{50}\NormalTok{)}
\CommentTok{\# We can create a probability matrix for each t. Lets pick t=10 years, so this is element 11 in each vector}
\ControlFlowTok{for}\NormalTok{(i }\ControlFlowTok{in} \DecValTok{1}\SpecialCharTok{:}\FunctionTok{nrow}\NormalTok{(p.matr))\{}
  \ControlFlowTok{for}\NormalTok{(j }\ControlFlowTok{in} \DecValTok{1}\SpecialCharTok{:}\FunctionTok{ncol}\NormalTok{(p.matr))\{}
    \ControlFlowTok{if}\NormalTok{(i }\SpecialCharTok{==}\NormalTok{ j) \{p.matr[i,j] }\OtherTok{\textless{}{-}} \FunctionTok{exp}\NormalTok{(}\SpecialCharTok{{-}}\FunctionTok{trans\_prob\_timelapse\_stay}\NormalTok{(full.df, i, }\AttributeTok{t=}\DecValTok{60}\NormalTok{, risks)}\SpecialCharTok{$}\NormalTok{baseline)[}\DecValTok{11}\NormalTok{]}
\NormalTok{    \} }\ControlFlowTok{else}\NormalTok{ \{p.matr[i,j] }\OtherTok{\textless{}{-}} \DecValTok{1}\SpecialCharTok{{-}}\FunctionTok{exp}\NormalTok{(}\SpecialCharTok{{-}}\FunctionTok{trans\_prob\_timelapse\_leave}\NormalTok{(full.df, i, j, }\AttributeTok{t=}\DecValTok{60}\NormalTok{, risks)}\SpecialCharTok{$}\NormalTok{baseline)[}\DecValTok{11}\NormalTok{]\}}
\NormalTok{  \}}
\NormalTok{\}}

\FunctionTok{apply}\NormalTok{(p.matr, }\DecValTok{1}\NormalTok{, sum)}
\end{Highlighting}
\end{Shaded}

\begin{verbatim}
## [1] 0.9700433 0.8509731 0.6353125 0.8537048 0.9697588 0.5580185 0.7973369
## [8] 0.9742572 0.5655932
\end{verbatim}

Which again seems incorrect. I also tried using the explicit integrated
Weibull hazards instead of using cumsum to approximate the integral, but
this does not solve the issue.

Something is going wrong in my analysis, but I do not really know where
or how to fix it. I'm not sure if the equations I use are correct, but
they at least resemble what they do in Alvares et al.~(2022). How to
enforce the probabilities sum to 1 is not clear to me.

\end{document}
